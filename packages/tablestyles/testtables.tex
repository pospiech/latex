%!TeX encoding=utf8
 
%% Dokumentenklasse (Koma Script) -----------------------------------------
\RequirePackage{fix-cm}
\documentclass[%
   %draft,     % draft mode (no images, layout errors shown)
   final,      % final mode 
%%% --- Paper Settings ---
   paper=a4,% [Todo: add alternatives]
   paper=portrait, % landscape
   pagesize=auto, % driver
%%% --- Base Font Size ---
   fontsize=11pt,%
%%% --- Koma Script Version ---
   version=last, %
%%% --- Global Package Options ---
   ngerman, % language (passed to babel and other packages)
            % (german, ngerman, english, french, ...)
 ]{scrartcl} % Classes: scrartcl, scrreprt, scrbook

\usepackage[T1]{fontenc}
\usepackage{lmodern}

\usepackage[utf8]{inputenc}

\usepackage{filecontents}

%%% Doc: xcolor.pdf
% Colors
\usepackage[
  dvipsnames, % Load a set of predefined colors 
  table,      % Load the colortbl package
  % fixpdftex,  % Load the pdfcolmk package (may be problematic)
  hyperref,   % Support  the  hyperref  package
  fixinclude, %Prevent dvips color reset before .eps file inclusion
]{xcolor}

%%% Doc: booktabs.pdf
\usepackage{booktabs}

\usepackage{ltxtable}

% tabu
\usepackage{tabu}
  \tabulinesep=5pt

\usepackage{float}

\usepackage{tablestyles}

\listfiles

%%% Document Start %%%%%%%%%%%%%%%%%%%%%%%%%%%%%%%%%%%%%%%%%%%%%%%%%%%%%%%%%%
\begin{document}
%%%%%%%%%%%%%%%%%%%%%%%%%%%%%%%%%%%%%%%%%%%%%%%%%%%%%%%%%%%%%%%%%%%%%%%%%%%%%

\begin{table}[H]
\tablestyle
%\begin{tabular}{|l|c|r|p{0.3\linewidth}|}
\begin{tabular}{|l|l|l|p{0.3\linewidth}|}
\theadstart
\thead \centering header l &
\thead \centering header c &   
\thead \centering header r &   
\thead \centering header p \tabularnewline
\tbody
left & center & right & 
text which is considerably longer than the width of the column \\
left & center & right & 
text which is considerably longer than the width of the column \\
left & center & right & 
text which is considerably longer than the width of the column \\
\tend
\end{tabular}
\caption{standard column types (lcrp)}
\label{tab:ct:standard}
\end{table}


\begin{table}[H]
%\tablestyle[sansbold]
\begin{tabu} to 0.8\textwidth
   {X[1,L]X[1,L]}
%\theadstart
\hline
\rowcolor{\tablecolor{head}}
\rowfont[l]{}%
%head & head &
 head & head \tabularnewline
%\tbody
\hline
%text which is considerably longer than the width of the column &
%text which is considerably longer than the width of the column &
text which is considerably longer than the width of the column &
text which is considerably longer than the width of the column \\
%text which is considerably longer than the width of the column &
%text which is considerably longer than the width of the column &
%text which is considerably longer than the width of the column &
%text which is considerably longer than the width of the column \\
%text which is considerably longer than the width of the column &
%text which is considerably longer than the width of the column &
%text which is considerably longer than the width of the column &
%text which is considerably longer than the width of the column \\
%\tend
\hline
\end{tabu}
\caption{standard column types (lcrp)}
\label{tab:ct:standard}
\end{table}


%
%\tableofcontents
%
%\section{tables styles}
%
%\subsection{table with no style applied}
%
%\begin{table}[H]
%\begin{tabular}{lll}
%header & header & header \tabularnewline
%content & content  & content \tabularnewline
%content & content  & content \tabularnewline
%content & content  & content \tabularnewline
%\end{tabular}
%\end{table}
%
%\subsection{table with no style and lines}
%
%\begin{table}[H]
%\begin{tabular}{lll}
%\hline
%header & header & header \tabularnewline
%\hline
%content & content  & content \tabularnewline
%content & content  & content \tabularnewline
%content & content  & content \tabularnewline
%\hline
%\end{tabular}
%\end{table}
%
%
%\subsection{table with lines in \emph{booktabs} style}
%
%\begin{table}[H]
%\begin{tabular}{lll}
%\toprule
%header & header & header \tabularnewline
%\midrule
%content & content  & content \tabularnewline
%content & content  & content \tabularnewline
%content & content  & content \tabularnewline
%\bottomrule
%\end{tabular}
%\end{table}
%
%
%\subsection{table with \emph{tablestyle} commands (default)}
%
%\begin{table}[H]
%\tablestyle
%\begin{tabular}{lll}
%\theadstart
%\thead header &
%\thead header &
%\thead header \tabularnewline
%\tbody
%content & content  & content \tabularnewline
%content & content  & content \tabularnewline
%content & content  & content \tabularnewline
%\tend
%\end{tabular}
%\end{table}
%%
%
%\begin{table}[H]
%\tablestyle
%\begin{tabular}{lll}
%\theadstart
%\thead header &
%\thead header &
%\thead header \tabularnewline
%\tbody
%content & content  & content \tabularnewline
%\tsubheadstart
%\tsubhead subhead &
%\tsubhead subhead &
%\tsubhead subhead \tabularnewline
%%
%content & content  & content \tabularnewline
%content & content  & content \tabularnewline
%\tend
%\end{tabular}
%\end{table}
%
%\subsection{table with \emph{tablestyle} commands (roman style)}
%
%\begin{table}[H]
%\tablestyle[roman]
%\begin{tabular}{lll}
%\theadstart
%\thead header &
%\thead header &
%\thead header \tabularnewline
%\tbody
%content & content  & content \tabularnewline
%%
%\tsubheadstart
%\tsubhead subhead &
%\tsubhead subhead &
%\tsubhead subhead \tabularnewline
%%
%content & content  & content \tabularnewline
%content & content  & content \tabularnewline
%\tend
%\end{tabular}
%\end{table}
%
%\subsection{table with tabularx (fixed width of table), sansbold style}
%
%%--------------------------------------------------------
%\begin{table}[H]
%\tablestyle[sansbold]
%\disablealternatecolors
%%
%\captionabove{table with tabularx (lXXlX), style sansbold, no alternating colors}
%\begin{tabularx}{\textwidth}{>{\itshape}lXXlX}
%\theadstart
%   \thead header &
%   \thead header &
%   \thead header &
%   \thead header &
%   \thead header \tabularnewline
%\tbody
%%
%description & content & content & content & content \tabularnewline
%description & content & content & content & content \tabularnewline
%description & content & content & content & content \tabularnewline
%\tend
%\end{tabularx}
%\end{table}
%
%%--------------------------------------------------------
%\begin{table}[H]
%\tablestyle[sansbold]
%%
%\captionabove{table with tabularx (lXXlX), style sansbold, with alternating colors and subheader}
%\begin{tabularx}{\textwidth}{>{\itshape}lXXlX}
%\theadstart
%   \thead header &
%   \thead header &
%   \thead header &
%   \thead header &
%   \thead header \tabularnewline
%\tbody
%%
%description & content & content & content & content \tabularnewline
%description & content & content & content & content \tabularnewline
%\tsubheadrow{5}{subheader} 
%description & content & content & content & content \tabularnewline
%description & content & content & content & content \tabularnewline
%description & content & content & content & content \tabularnewline
%\tend
%\end{tabularx}
%\end{table}
%
%
%\subsection{table with sansboldbw style}
%
%\begin{table}[H]
%\tablestyle[sansboldbw]
%\captionabove{table with darker alternating colors}
%\begin{tabular}{*{2}{v{0.45\textwidth}}}
%   \theadstart
%   \thead header &
%   \thead header \tabularnewline
%% Zwischenkopf
%\tbody
%\tsubheadrow{2}{subheader} 
% content  & content \tabularnewline
% content  & content \tabularnewline
% content  & content \tabularnewline
% content  & content \tabularnewline
% content  & content \tabularnewline
% content  & content \tabularnewline
%\tsubheadrow{2}{subheader} 
% content  & content \tabularnewline
% content  & content \tabularnewline
%\tend
%   \end{tabular}
%\end{table}
%
%\subsection{table across pages (longtable)}
%
%\begin{filecontents}{longtableexample}
%\begin{longtable}{>{\itshape}l*{5}{Y}}
%\captionabove{table across pages (longtable) with tabularx columns} \\
%   \theadstart
%   \thead header &
%   \thead header &
%   \thead header &
%   \thead header &
%   \thead header &
%   \thead header \tabularnewline
%\endfirsthead
%   \tbody 
%   \multicolumn{6}{r}{\emph{continued on next page \ldots{}}}
%\endfoot
%	\tend
%\endlastfoot
%\tbody
%description   & content & content & content & content & content \tabularnewline
%description   & content & content & content & content & content \tabularnewline
%description   & content & content & content & content & content \tabularnewline
%description   & content & content & content & content & content \tabularnewline
%description   & content & content & content & content & content \tabularnewline
%description   & content & content & content & content & content \tabularnewline
%description   & content & content & content & content & content \tabularnewline
%description   & content & content & content & content & content \tabularnewline
%description   & content & content & content & content & content \tabularnewline
%description   & content & content & content & content & content \tabularnewline
%description   & content & content & content & content & content \tabularnewline
%description   & content & content & content & content & content \tabularnewline
%description   & content & content & content & content & content \tabularnewline
%description   & content & content & content & content & content \tabularnewline
%description   & content & content & content & content & content \tabularnewline
%description   & content & content & content & content & content \tabularnewline
%description   & content & content & content & content & content \tabularnewline
%description   & content & content & content & content & content \tabularnewline
%description   & content & content & content & content & content \tabularnewline
%description   & content & content & content & content & content \tabularnewline
%description   & content & content & content & content & content \tabularnewline
%description   & content & content & content & content & content \tabularnewline
%description   & content & content & content & content & content \tabularnewline
%description   & content & content & content & content & content \tabularnewline
%description   & content & content & content & content & content \tabularnewline
%description   & content & content & content & content & content \tabularnewline
%description   & content & content & content & content & content \tabularnewline
%description   & content & content & content & content & content \tabularnewline
%description   & content & content & content & content & content \tabularnewline
%description   & content & content & content & content & content \tabularnewline
%description   & content & content & content & content & content \tabularnewline
%description   & content & content & content & content & content \tabularnewline
%description   & content & content & content & content & content \tabularnewline
%description   & content & content & content & content & content \tabularnewline
%description   & content & content & content & content & content \tabularnewline
%description   & content & content & content & content & content \tabularnewline
%description   & content & content & content & content & content \tabularnewline
%description   & content & content & content & content & content \tabularnewline
%description   & content & content & content & content & content \tabularnewline
%description   & content & content & content & content & content \tabularnewline
%description   & content & content & content & content & content \tabularnewline
%description   & content & content & content & content & content \tabularnewline
%description   & content & content & content & content & content \tabularnewline
%description   & content & content & content & content & content \tabularnewline
%description   & content & content & content & content & content \tabularnewline
%description   & content & content & content & content & content \tabularnewline
%description   & content & content & content & content & content \tabularnewline
%description   & content & content & content & content & content \tabularnewline
%description   & content & content & content & content & content \tabularnewline
%description   & content & content & content & content & content \tabularnewline
%description   & content & content & content & content & content \tabularnewline
%description   & content & content & content & content & content \tabularnewline
%\end{longtable}
%\end{filecontents}
%
%{
%   \tablestyle[sansbold]
%   \LTXtable{\textwidth}{longtableexample.tex}
%}
%
%
%\subsection{colored vertical lines}
%
%\begin{table}[H]
%\tablestyle
%\begin{tabularx}{0.8\textwidth}{%
%l!{\color[gray]{0.7}\vline}
%CC!{\color[gray]{0.7}\vline}
%CC!{\color[gray]{0.7}\vline}
%C
%}
%\theadstart
%header & 
%\multicolumn{2}{>{\columncolor{\tablecolor{head}}\thead}c!{\color[gray]{0.7}\vline}}
%{header} & 
%\multicolumn{2}{>{\columncolor{\tablecolor{head}}\thead}c!{\color[gray]{0.7}\vline}}
%{header} & 
%\multicolumn{1}{>{\columncolor{\tablecolor{head}}\thead}c}
%{header} 
%%
%\tabularnewline
%\tbody
%%
%description & 0,3 & 0,35 & 0,5 & 0,65 & 0,80\tabularnewline
%%
%description & 0,3 & 0,35 & 0,5 & 0,65 & 0,80\tabularnewline
%%
%description & 0,3 & 0,35 & 0,5 & 0,65 & 0,80\tabularnewline
%%
%description & 0,3 & 0,35 & 0,5 & 0,65 & 0,80\tabularnewline
%\tend
%\end{tabularx}
%\end{table}
%
%\section{column types}
%
%\subsection{standard table}
%%--------------------------------------------------------
%\begin{table}[H]
%\tablestyle
%%
%\captionabove{standard table (lcr)}
%\begin{tabular}{|l|c|r|}
%\theadstart
%   \thead \centering header l &
%   \thead \centering header c &   
%   \thead \centering header r \tabularnewline
%\tbody
%%
%left & center & right \tabularnewline
%\tend
%\end{tabular}
%\end{table}
%
%\subsection{fixed width colummns}
%%--------------------------------------------------------
%\begin{table}[H]
%\tablestyle
%%
%\captionabove{table with tabular (pmb)}
%\begin{tabular}{|p{0.28\textwidth}|m{0.28\textwidth}|b{0.28\textwidth}|}
%\theadstart
%   \thead \centering header p &
%   \thead \centering header m &   
%   \thead \centering header b \tabularnewline
%\tbody
%%
%The \LaTeX{} document preparation 
%system is a special version of Donald
%Knuth's \TeX{} program. \TeX{} is a 
%sophisticated program designed to 
%produce high-quality typesetting, 
%especially for mathematical text.
%&
%The \LaTeX{} document preparation 
%system is a special version of Donald
%Knuth's \TeX{} program. \TeX{} is a 
%sophisticated program designed to 
%produce high-quality typesetting, 
%especially for mathematical text.
%&
%The \LaTeX{} document preparation 
%system is a special version of Donald
%Knuth's \TeX{} program. \TeX{} is a 
%sophisticated program designed to 
%produce high-quality typesetting, 
%especially for mathematical text.
%\tabularnewline
%\tend
%\end{tabular}
%\end{table}
%
%\subsection{table with tabularx und left, centered and right columns}
%%--------------------------------------------------------
%\begin{table}[H]
%\tablestyle
%%
%\captionabove{table with tabularx (XCY)}
%\begin{tabularx}{\textwidth}{|X|C|Y|}
%\theadstart
%   \thead \centering header X &
%   \thead \centering header C &   
%   \thead \centering header Y \tabularnewline
%\tbody
%%
%The \LaTeX{} document preparation 
%system is a special version of Donald
%Knuth's \TeX{} program. \TeX{} is a 
%sophisticated program designed to 
%produce high-quality typesetting, 
%especially for mathematical text.
%&
%The \LaTeX{} document preparation 
%system is a special version of Donald
%Knuth's \TeX{} program. \TeX{} is a 
%sophisticated program designed to 
%produce high-quality typesetting, 
%especially for mathematical text.
%&
%The \LaTeX{} document preparation 
%system is a special version of Donald
%Knuth's \TeX{} program. \TeX{} is a 
%sophisticated program designed to 
%produce high-quality typesetting, 
%especially for mathematical text.
%\tabularnewline
%\tend
%\end{tabularx}
%\end{table}
%
%\subsection{table with tabularx and fixed columns}
%%--------------------------------------------------------
%\begin{table}[H]
%\tablestyle
%%
%\captionabove{table with tabularx and fixed columns (vXX) using improved ragged right}
%\begin{tabularx}{\textwidth}{|v{0.2\textwidth}|X|X|}
%\theadstart
%   \thead \centering header v &
%   \thead \centering header X &   
%   \thead \centering header X \tabularnewline
%\tbody
%%
%The \LaTeX{} document preparation 
%system is a special version of Donald
%Knuth's \TeX{} program. \TeX{} is a 
%sophisticated program designed to 
%produce high-quality typesetting, 
%especially for mathematical text.
%&
%The \LaTeX{} document preparation 
%system is a special version of Donald
%Knuth's \TeX{} program. \TeX{} is a 
%sophisticated program designed to 
%produce high-quality typesetting, 
%especially for mathematical text.
%&
%The \LaTeX{} document preparation 
%system is a special version of Donald
%Knuth's \TeX{} program. \TeX{} is a 
%sophisticated program designed to 
%produce high-quality typesetting, 
%especially for mathematical text.
%\tabularnewline
%\tend
%\end{tabularx}
%\end{table}
%
%\section{itemize in tables}
%
%\begin{table}[H]
%\tablestyle
%%
%\captionabove{table item list using \texttt{tableitemize}}
%\begin{tabularx}{\textwidth}{|v{0.2\textwidth}|X|X|}
%\theadstart
%   \thead \centering header v &
%   \thead \centering header items (X) &   
%   \thead \centering header X \tabularnewline
%\tbody
%%
%The \LaTeX{} document preparation 
%system is a special version of Donald
%Knuth's \TeX{} program. \TeX{} is a 
%sophisticated program designed to 
%produce high-quality typesetting, 
%especially for mathematical text.
%&
%\tableitemize
%\begin{itemize}
%\item The \LaTeX{} document preparation 
%system is a special version of Donald
%Knuth's \TeX{} program.
%%
%\item \TeX{} is a 
%sophisticated program designed to 
%produce high-quality typesetting,
%%
%\item especially for mathematical text.
%\end{itemize}
%&
%The \LaTeX{} document preparation 
%system is a special version of Donald
%Knuth's \TeX{} program. \TeX{} is a 
%sophisticated program designed to 
%produce high-quality typesetting, 
%especially for mathematical text.
%\tabularnewline
%\tend
%\end{tabularx}
%\end{table}


%%% Dokument ENDE %%%%%%%%%%%%%%%%%%%%%%%%%%%%%%%%%%%%%%%%%%%%%%%%%%%%%%%%%%
\end{document}
%%%%%%%%%%%%%%%%%%%%%%%%%%%%%%%%%%%%%%%%%%%%%%%%%%%%%%%%%%%%%%%%%%%%%%%%%%%%
--