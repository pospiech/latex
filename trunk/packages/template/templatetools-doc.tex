\documentclass{ltxdoc}
\EnableCrossrefs
\CodelineIndex
\RecordChanges
%
\usepackage{cmap} 
\usepackage[T1]{fontenc} % T1 Schrift Encoding
\usepackage{lmodern}
%
\RequirePackage{enumitem}
\RequirePackage[table]{xcolor}

\colorlet{lstcolorStringLatex}{green!40!black!100}
\colorlet{lstcolorCommentLatex}{green!50!black!100}
\definecolor{lstcolorKeywordLatex}{rgb}{0,0.47,0.80}

\lstdefinestyle{lstStyleLaTeX}{
   ,style=lstStyleBase
%%% colors
   ,stringstyle=\color{lstcolorStringLatex}%
   ,keywordstyle=\color{lstcolorKeywordLatex}%
   ,commentstyle=\color{lstcolorCommentLatex}%
   ,% backgroundcolor=\color{codebackcolor}%
%%% Frames
   ,frame=single%
   ,%frameround=tttt%
   ,%framesep = 10pt%
   ,%framerule = 0pt%
   ,rulecolor = \color{black}%
%%% language
   ,language = [LaTeX]TeX%
   ,morekeywords={maketitle,tableofcontents,subsection,text,includegraphics,cha
pter,tableofcontents,section,subsection,subsubsection,paragraph,si,SI,textmu,newcolumntype,rowcolor,rowcolors,bottomrule,toprule,midrule,mainmatter,frontmatter,geometry,KOMAoptions,newblock,ExecuteBibliographyOptions,addbibresource,operatorname,dfrac,DeclareMathOperator,micro,sisetup,captionof,listoffigures,listoftables,mathcal,underset}%
}

\lstloadlanguages{[LaTeX]TeX}

%
\usepackage[
  ,backref=page%
  ,pagebackref=false%
  ,hyperindex=true%
  ,hyperfootnotes=false%
  ,bookmarks=true%
  ,pdfpagelabels=true%
]{hyperref}
%
\usepackage[]{bookmark}
%
\definecolor{pdfanchorcolor}{named}{black}
\definecolor{pdfmenucolor}{named}{red}
\definecolor{pdfruncolor}{named}{cyan}
\definecolor{pdfurlcolor}{rgb}{0,0,0.6}
\definecolor{pdffilecolor}{rgb}{0.7,0,0}
\definecolor{pdflinkcolor}{rgb}{0,0,0.6}
\definecolor{pdfcitecolor}{rgb}{0,0,0.6}
%
\hypersetup{
	,draft=false, % all hypertext options are turned off
	,final=true   % all hypertext options are turned on
	,debug=false  % extra diagnostic messages are printed in the log file
	,hypertexnames=true % use guessable names for links
	,naturalnames=false % use LATEX-computed names for links
	,setpagesize=true   % sets page size by special driver commands
	,raiselinks=true    % forces commands to reflect the real height of the link 
	,breaklinks=true    % Allows link text to break across lines
	,pageanchor=true    % Determines whether every page is given an implicit
	,plainpages=false   % Forces page anchors to be named by the arabic
	,linktocpage=true   % make page number, not text, be link on TOC, LOF and LOT
	,colorlinks=true    % Colors the text of links and anchors.
	,linkcolor  =pdflinkcolor   % Color for normal internal links.
	,anchorcolor=pdfanchorcolor % Color for anchor text.
	,citecolor  =pdfcitecolor   % Color for bibliographical citations in text.
	,filecolor  =pdffilecolor   % Color for URLs which open local files.
	,menucolor  =pdfmenucolor   % Color for Acrobat menu items.
	,runcolor   =pdfruncolor    % Color for run links (launch annotations).
	,urlcolor   =pdfurlcolor    % color magenta Color for linked URLs.
	,bookmarksopen=true     % If Acrobat bookmarks are requested, show them
	,bookmarksopenlevel=2   % level (\maxdimen) to which bookmarks are open
	,bookmarksnumbered=true %
	,bookmarkstype=toc      %
	,pdfpagemode=UseOutlines %
	,pdfstartpage=1         % Determines on which page the PDF file is opened.
	,pdfstartview=FitH      % Set the startup page view
	,pdfremotestartview=Fit % Set the startup page view of remote PDF files
	,pdfcenterwindow=false  %
	,pdffitwindow=false     % resize document window to fit document size
	,pdfnewwindow=false     % make links that open another PDF file 
	,pdfdisplaydoctitle=true  % display document title instead of file name 
} % end: hypersetup
%

\newcommand{\package}[1]{\texttt{#1}}

\listfiles
\begin{document}

\changes{0.1}{2011/10/01}{Initial version.}

\DoNotIndex{\newcommand,\newenvironment}

\providecommand*{\url}{\texttt}
\title{The \textsf{templatetools} package}
\author{Matthias Pospiech \\ \url{matthias.pospiech@gmx.de}}
\date{\fileversion~from \filedate}

\maketitle
\begin{abstract}\noindent
Collection of tools, which are helpful for the creation of a \LaTeX{} template if conditional paths for code execution are required. Most of them are already
available from other packages, so that this package only provides a wrapper
with a unique code style. 
\end{abstract}
\tableofcontents

\section{Commands}

\subsection{if command sequence defined}
The following commands are all wrappers to \cs{ifcsdef} with only mandatory options:
\begin{itemize}
\item\cs{IfDefined}\marg{command}\marg{defined}
\item\cs{IfUndefined}\marg{command}\marg{undefined}
\item\cs{IfElseDefined}\marg{command}\marg{defined}\marg{undefined}
\item\cs{IfElseUndefined}\marg{command}\marg{undefined}\marg{defined}
\end{itemize}

\begin{lstlisting}[style=demostyle]
% Requires: Command \upmu
\IfDefined{upmu}{\usepackage[upmu]{gensymb}}
\end{lstlisting}

\subsection{if package(s) loaded}

The following commands are available in the preamble and in the document
\begin{itemize}
\item\cs{IfPackageLoaded}\marg{package}\marg{is loaded}
\item\cs{IfPackageNotLoaded}\marg{package}\marg{is not loaded}
\item\cs{IfElsePackageLoaded}\marg{package}\marg{loaded}\marg{not loaded}
\item\cs{IfPackagesLoaded}\marg{package1, package2, ... }
   \marg{all packages are loaded}
\item\cs{IfPackagesNotLoaded}\marg{package1, package2, ... }
   \marg{all packages are not loaded}
\end{itemize}

\begin{lstlisting}[style=demostyle]
% Incompatible: gensymb, units
\IfPackagesNotLoaded{gensymb, units}
  \usepackage{siunitx}
}
\end{lstlisting}

The code for \cs{IfPackagesLoaded} and the negation is based on the solution of ?? of this question \url{tx}. All others are wrappers to the commands from the package \package{ltxcmds}.


\subsection{if draft}

The following commands are available in the preamble and in the document:
\begin{itemize}
\item\cs{IfDraft}\marg{in draft mode}
\item\cs{IfNotDraft}\marg{not in draft mode}
\item\cs{IfNotDraftElse}\marg{in draft mode}\marg{not in draft mode}
\end{itemize}

\begin{lstlisting}[style=demostyle]
% print text of every \index{entry} to the margin
\IfDraft{
  \usepackage{showidx}
}
\end{lstlisting}

\subsection{execute after package}

The command 
\begin{itemize}
\item\cs{ExecuteAfterPackage}\marg{after this package}\marg{execute this code}
\end{itemize}
is simply a wrapper to \cs{AfterPackage} from \package{scrlfile} (Koma Script).

\begin{lstlisting}[style=demostyle]
% -> load afer hyperref 
\ExecuteAfterPackage{hyperref}{\usepackage{ltxtable}}
\end{lstlisting}

\subsection{if column type defined}
\LaTeX{} provides no tool to check for the existence of a column type. This is provided by the following command:
\begin{itemize}
\item\cs{IfColumntypeDefined}\marg{columntype character}\marg{is defined}\marg{is undefined}
\end{itemize}

\subsection{template definitions}

The following commands can be use to define macros which can then be used later.  
\begin{itemize}
\item\cs{SetTemplateDefinition}\marg{Group}\marg{Property}\marg{Code}
\item\cs{UseDefinition}\marg{Group}\marg{Property} 
\end{itemize}
%
In the following example this allows to switch the colors anywhere in the document:

\begin{lstlisting}[style=demostyle]
\SetTemplateDefinition{Target}{Web}{%
  \definecolor{pdfurlcolor}{rgb}{0,0,0.6}
}%
\SetTemplateDefinition{Target}{Print}{%
  \definecolor{pdfurlcolor}{rgb}{0,0,0}
}%
% Apply colors for web
\UseDefinition{Target}{Web}
\end{lstlisting}
\end{document}
