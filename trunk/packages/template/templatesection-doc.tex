\documentclass{ltxdoc}
\EnableCrossrefs
\CodelineIndex
\RecordChanges
%
\usepackage{cmap} 
\usepackage[T1]{fontenc} % T1 Schrift Encoding
\usepackage{lmodern}

\RequirePackage{templatesection}%
\RequirePackage{listings}
\RequirePackage[table]{xcolor}
\RequirePackage{enumitem}

\colorlet{lstcolorStringLatex}{green!40!black!100}
\colorlet{lstcolorCommentLatex}{green!50!black!100}
\definecolor{lstcolorKeywordLatex}{rgb}{0,0.47,0.80}

\lstdefinestyle{lstStyleLaTeX}{
   ,style=lstStyleBase
%%% colors
   ,stringstyle=\color{lstcolorStringLatex}%
   ,keywordstyle=\color{lstcolorKeywordLatex}%
   ,commentstyle=\color{lstcolorCommentLatex}%
   ,% backgroundcolor=\color{codebackcolor}%
%%% Frames
   ,frame=single%
   ,%frameround=tttt%
   ,%framesep = 10pt%
   ,%framerule = 0pt%
   ,rulecolor = \color{black}%
%%% language
   ,language = [LaTeX]TeX%
   ,morekeywords={maketitle,tableofcontents,subsection,text,includegraphics,cha
pter,tableofcontents,section,subsection,subsubsection,paragraph,si,SI,textmu,newcolumntype,rowcolor,rowcolors,bottomrule,toprule,midrule,mainmatter,frontmatter,geometry,KOMAoptions,newblock,ExecuteBibliographyOptions,addbibresource,operatorname,dfrac,DeclareMathOperator,micro,sisetup,captionof,listoffigures,listoftables,mathcal,underset}%
}

\lstloadlanguages{[LaTeX]TeX}

%
\usepackage[
  ,backref=page%
  ,pagebackref=false%
  ,hyperindex=true%
  ,hyperfootnotes=false%
  ,bookmarks=true%
  ,pdfpagelabels=true%
]{hyperref}
%
\usepackage[]{bookmark}
%
\definecolor{pdfanchorcolor}{named}{black}
\definecolor{pdfmenucolor}{named}{red}
\definecolor{pdfruncolor}{named}{cyan}
\definecolor{pdfurlcolor}{rgb}{0,0,0.6}
\definecolor{pdffilecolor}{rgb}{0.7,0,0}
\definecolor{pdflinkcolor}{rgb}{0,0,0.6}
\definecolor{pdfcitecolor}{rgb}{0,0,0.6}
%
\hypersetup{
	,draft=false, % all hypertext options are turned off
	,final=true   % all hypertext options are turned on
	,debug=false  % extra diagnostic messages are printed in the log file
	,hypertexnames=true % use guessable names for links
	,naturalnames=false % use LATEX-computed names for links
	,setpagesize=true   % sets page size by special driver commands
	,raiselinks=true    % forces commands to reflect the real height of the link 
	,breaklinks=true    % Allows link text to break across lines
	,pageanchor=true    % Determines whether every page is given an implicit
	,plainpages=false   % Forces page anchors to be named by the arabic
	,linktocpage=true   % make page number, not text, be link on TOC, LOF and LOT
	,colorlinks=true    % Colors the text of links and anchors.
	,linkcolor  =pdflinkcolor   % Color for normal internal links.
	,anchorcolor=pdfanchorcolor % Color for anchor text.
	,citecolor  =pdfcitecolor   % Color for bibliographical citations in text.
	,filecolor  =pdffilecolor   % Color for URLs which open local files.
	,menucolor  =pdfmenucolor   % Color for Acrobat menu items.
	,runcolor   =pdfruncolor    % Color for run links (launch annotations).
	,urlcolor   =pdfurlcolor    % color magenta Color for linked URLs.
	,bookmarksopen=true     % If Acrobat bookmarks are requested, show them
	,bookmarksopenlevel=2   % level (\maxdimen) to which bookmarks are open
	,bookmarksnumbered=true %
	,bookmarkstype=toc      %
	,pdfpagemode=UseOutlines %
	,pdfstartpage=1         % Determines on which page the PDF file is opened.
	,pdfstartview=FitH      % Set the startup page view
	,pdfremotestartview=Fit % Set the startup page view of remote PDF files
	,pdfcenterwindow=false  %
	,pdffitwindow=false     % resize document window to fit document size
	,pdfnewwindow=false     % make links that open another PDF file 
	,pdfdisplaydoctitle=true  % display document title instead of file name 
} % end: hypersetup
%
%%% Set document layout / variables
\setlength{\parindent}{0pt}
\setlength{\parskip}{0.5\baselineskip}
\setcounter{secnumdepth}{2}
\setcounter{tocdepth}{2}
%%% doc commands
\newcommand{\package}[1]{\texttt{#1}}
\newcommand{\option}[1]{\texttt{#1}}
\newcommand{\parameter}[1]{\texttt{#1}}
\renewcommand\arg[1]{\meta{\normalfont\slshape#1}}
% 
\newcommand{\latex}{\LaTeX}
\newcommand{\Default}[1]{\par Default: \texttt{#1} \par}
\newcommand{\Example}[1]{\par Example: \texttt{#1} \par}

\newenvironment{Optionlist}{%
\begin{flushright}%
%  Style  changes
\small\renewcommand{\arraystretch}{1.4}%
%  tabu
\begin{tabu} to 1.0\linewidth  {>{\ttfamily}l<{\normalfont}X[1,l]}%
}{%
\end{tabu}%
\end{flushright}%
}


\listfiles
\begin{document}

\changes{0.1}{2011/01/20}{Initial version.}

\DoNotIndex{\newcommand,\newenvironment}

\providecommand*{\url}{\texttt}
\title{The \textsf{templatesection} package}
\author{Matthias Pospiech \\ \url{matthias.pospiech@gmx.de}}
\date{0.1~from \filedate}


\maketitle
\section{Introduction}
This packages provides an environment to switch a section of code
on or off. The code can be placed anywhere in the file and is 
not limited to the document or the preamble. 
The motivation for this package was to have commands which allow to preselect
sections of code in a preamble of a template, which are executed or not.

\section{Origin of the code}
The code is based on the \texttt{verbatim.sty} package and was 
originally modified by Ulrich Diez to match the pure comment
functionality. Further modifications are contributed by 
Matthias Pospiech. 
During the development some discussion about the best approach took place
on de.comp.text.tex (\url{http://groups.google.com/group/de.comp.text.tex/browse_thread/thread/2c18f0c221ab167f/}), which resulted in the current code. 
 
\section{Usage}
The idea of the following commands is to define a collection of code, here notated as a \emph{section}, which can be executed as it would be without the commands or which is not executed at all. To use that section it must be defined with \emph{true} (execute code) or \emph{false} (skip code).

\DescribeMacro{\DefineTemplateSection} \oarg{true/false}\marg{name} 
Defines a code section with a \emph{name}. The default is \emph{true}, thus the code will be executed.

\DescribeMacro{\SetTemplateSection} \marg{name}\marg{true/false} is like \cs{DefineTemplateSection}, but with both arguments mandatory.

\DescribeMacro{\BeginTemplateSection} \marg{name} starts the code section with the given name and

\DescribeMacro{\EndTemplateSection} \marg{name} ends the code section with the given name. Note that both commands need to be paired and not to be nested with other code sections.

\cs{BeginTemplateSection} and \cs{EndTemplateSection} mimic an environment. It would be preferable to define them as an environment, but that opens a group in \TeX{}, which has many disadvantages. For example this would make it impossible to load packages. Therefore this package defines paired commands and consequently, has no such limitations.

%
\section{Example}
In the following code the first section is going to be executed and the second and the third are completely skipped.
\begin{lstlisting}[style=demostyle]
\DefineTemplateSection[true]{ExecuteMe}
\DefineTemplateSection[false]{SkipMe}
%
\BeginTemplateSection{ExecuteMe}
  This sentence has ...
\EndTemplateSection{ExecuteMe}
%
\BeginTemplateSection{SkipMe}
  no end.
\EndTemplateSection{SkipMe}
%
\SetTemplateSection{ExecuteMe}{false}
%
\BeginTemplateSection{ExecuteMe}
  a different ending.
\EndTemplateSection{ExecuteMe}
\end{lstlisting}

\DefineTemplateSection[true]{ExecuteMe}
\DefineTemplateSection[false]{SkipMe}
\BeginTemplateSection{ExecuteMe}
  This sentence has ...
\EndTemplateSection{ExecuteMe}
%
\BeginTemplateSection{SkipMe}
  no end.
\EndTemplateSection{SkipMe}
%
\SetTemplateSection{ExecuteMe}{false}
%
\BeginTemplateSection{ExecuteMe}
  a different ending.
\EndTemplateSection{ExecuteMe}

\end{document}
