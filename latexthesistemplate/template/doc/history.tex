% !TeX encoding=utf8
% !TeX spellcheck = en-US

\chapter{Changes and history}
\label{appendix:doc:changes}

\minisec{Version numbers}
The version number is defined by the KOMA-Script version followed by the template version. 
Version 3.2.0 is thus a huge change from 3.1.0 with both compatible for version 3 of KOMA-script.
%This was not intentionally defined this way. It just happened that with the version 3.x of KOMA-script
%a new template version was released that was the third one and later there was little reason
%to change this version number scheme.


\minisec{2018/09 v3.2.5} %
Minor bug fixes and other changes
\begin{itemize}
	\item The template failed to compile the \package{biblatex} code in the publication list correct. This was due to a change in the code base of \package{biblatex}.
	\item Examples for \package{glossaries} were added to the template.
	\item replaced \package{scrpage2} by \package{scrlayer-scrpage}
\end{itemize}

\minisec{2015/08 v3.2.4} %
Minor bug fixes and other changes
\begin{itemize}
\item The template failed to compile with latest package \package{titlesec} in combination with KOMA-script. Since both are not compatible and can be used only with workarounds within KOMA-Script the package \package{titlesec} was removed and the style changes applied using different commands.
\end{itemize}


\minisec{2015/08 v3.2.3} %
Minor bug fixes and other changes
\begin{itemize}
\item The template failed to compile with TeX Live 2015. Package \package{pageslts} requires \package{atveryend} to be loaded before \package{etoolbox}. 
\item Removed package \package{fixltx2e}
\end{itemize}


\minisec{2014/07 v3.2.2} %
Bug fixes, Improvements and other changes
\begin{itemize}
\item The template failed to compile with TeX Live 2014. The error was in the definition of \cs{addmoretexcs}. 
\item The options of \package{geometry} were not well thought out. If a spacing factor was introduced this could lead to an ugly page layout. All options of \package{geometry} are now such that the page layout is similar to the one of \package{typearea} with DIV12. 
\item The publications lists are now bibliography lists create with \cs{printbibliography}. Previously these needed to be created completely manual.
\item New magic comment for the bibliography tool added.
\item Removed packages. These are now available from CTAN or better the distribution package manager. 
\end{itemize}

\minisec{2014/01 v3.2.1} %
Mainly enhancements and bug fixing. The following list is a selection:
\begin{itemize}
\item Selection of packages for the ``no room for a new \textbackslash{}write´´ problem added.  
\item Update of glossary lists handling. New file for definitions and update of \texttt{glossaries} options.
\item Added \package{tocstyle} to the list of used packages.
\item Added file list with date of release
\item Enabled \package{typearea} instead of \package{geometry}. This was basically a mistake in the code.
\end{itemize}
%
\minisec{2013/06 v3.2.0} %
Initial Release of the complete reworked template with several outstanding features and changes:
\begin{itemize}
\item Complete new compilation of packages (up to date at 2013) with framework for selecting package sections. 
\item Focus on a target group of user who want to write thesis like documents.
\item Introduction of a template documentation.
\item Significant enhancements in the latex examples. 
It transformed from a simple rudimentary test and sample document to a test and example framework
with examples for every package.
\item Translation of all texts and comments into English. It targets therefor a much broader audience.
\end{itemize}
%
\minisec{2008/12 v3.1.0 (LaTeX-Vorlage 3)} %
New release due to a rework for KOMA-Script 3.x. 
The basic design was adopted from the previous version.
Further changes mainly in terms of package updates and bug fixes.

\minisec{2006/06 v2.0.0 (LaTeX-Vorlage)}
Initial online release of the template. It is based on KOMA-Script 2.x,
supports most modern packages (at year 2006), provides most package options in the code
and a documentation of the preamble code. The basic language is German.
Additionally it provides a demo file for testing and showing the document layout.
%
